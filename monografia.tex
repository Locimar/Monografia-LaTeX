%% monografia.tex                                                    %
%% Copyright 2013 Daniel Girotto                                     %
%                                                                    %
% This work may be distributed and/or modified under the             %
% conditions of the LaTeX Project Public License, either version 1.3 %
% of this license or (at your option) any later version.             %
% The latest version of this license is in                           %
%   http://www.latex-project.org/lppl.txt                            %
% and version 1.3 or later is part of all distributions of LaTeX     %
% version 2005/12/01 or later.                                       %
%                                                                    %
% This work has the LPPL maintenance status `maintained'.            %
%                                                                    %
% The Current Maintainer of this work is Daniel Girotto              %
% <danielcgirotto@gmail.com>.                                        %    

% ------------------------------------------------------------ %
% Monografia                                                   %
% ------------------------------------------------------------ %
\documentclass[pnumromarab, normaltoc, a4paper, 12pt]{abnt}

\usepackage{url}
\usepackage{color}
\usepackage{leading}
\usepackage{lmodern}
\usepackage{graphicx}
\usepackage{listings}
\usepackage{abnt-unochapeco}
\usepackage[abnt-emphasize=bf]{abnt-alf}
\usepackage{varwidth}
\usepackage{times}
\usepackage[utf8]{inputenc}
\usepackage[brazil]{babel}

\begin{document}

\newcommand{\area}{área de ciências exatas e ambientais}
\newcommand{\curso}{ciência da computação}
\newcommand{\grau}{bacharelado}
\newcommand{\grautitulo}{bacharel}
\newcommand{\supervisortcc}{Prof. Sandro Silva de Oliveira, Me.}
\newcommand{\coordenador}{Profa. Viviane Duarte Bonfim, Ma.}
\newcommand{\membrobancaa}{Prof. Marcos Antonio Moretto, Esp.}
\newcommand{\membrobancab}{Prof. Jean Carlos Hennrichs, Esp.}
\newcommand{\datadefesa}{25 junho de 2013}
\newcommand{\textofolhaaprovacao}{Este trabalho de conclusão de curso foi
   julgado adequado para obtenção de título de graduação em \datadefesa, e foi
   aprovado pelo curso de \grau\  em \curso\ da \instituicao}

\renewcommand{\autor}{Daniel Carlos Girotto}
\renewcommand{\instituicao}{Universidade Comunitária da Região de Chapecó}
\newcommand{\instituicaosimples}{Unochapecó}
\renewcommand{\titulo}{Aplicativo Android para o National Center for
  Biotechnology Information}
\renewcommand{\local}{Chapecó}
\renewcommand{\data}{\today}
\renewcommand{\comentario}{Relatório do Trabalho de Conclusão de Curso submetido
  à Universidade \mbox{Comunitária} da Região de Chapecó para obtenção do título
  de bacharel em Ciência da Computação.}
\renewcommand{\orientador}{Prof. Marcelo Cezar Pinto, Me.}

\capa
\folhaderosto
\folhadeaprovacao

% \dedicatoria{Dedico.....\\
%   Item OPCIONAL, deve ficar posicionado ao final da folha.\\ É uma menção onde o
%   autor presta homenagem ou dedica o trabalho a alguém}
% 
% \agradecimento{Agradeço....\\
%   Item OPCIONAL, deve ficar posicionado ao final da folha.\\ São menções a
%   pessoas e/ou instituições das quais eventualmente recebeu apoio e que
%   concorreram de maneira relevante para o desenvolvimento do trabalho.}
% 
% \epigrafe{Epígrafe\\
%   Item OPCIONAL, deve ficar posicionado ao final da folha.\\ É a inscrição de um
%   trecho em prosa ou composição poética que de certa forma embasou a construção
%   do trabalho, seguida da indicação de autoria.\\ Fulano de tal}

% \resumo {Conteúdo do resumo} {Palavras-Chave}
\resumo{É a apresentação concisa do texto, destacando seus aspectos de  maior
  relevância. É redigido em um único parágrafo, contendo entre 250 e 500
  palavras. Usar terceira pessoa do singular. Não usar citações bibliográficas.
  Ressaltar objetivos, métodos, resultados e conclusões do trabalho.}
{entre três e cinco palavras-chave, separadas por ponto e  vírgula.}

% \abstract {Abstract content} {Keywords}
\abstract{It is a brief presentation of the text in English, where the most
  relevant aspects are underlined. It should be written in the third person
  singular, in an only paragraph containing from 250 to 500 words. You should
  not mention references. It is important to write the objectives, method,
  results and the final remarks of the study.}
{from 3 to 5 keywords, separated by semicolon.}

% ------------------------------------------------------------ %
% Altera o título do Sumário                                   %
% ------------------------------------------------------------ %
\renewcommand{\contentsname}%
  {\vspace*{.8cm}
  \normalsize{\bfseries\MakeUppercase{sumário}}%
  \vspace{1.4cm}}

\sumario

\bibliography{referencias}

\end{document}
